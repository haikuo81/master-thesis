\chapter{Conclusion}

The goal of the suitability analysis of Kubernetes for Seznam.cz was to become familiar with the basics such as using Docker and virtualization in general. That’s why the beginning of this thesis is dedicated to the theory about Docker containers and using Kubernetes for orchestrating them. 

The next step was to employ those information in Seznam.cz’s specific environment. I~mapped and examined the current situation of application development and deployment at Seznam.cz and was wondering about how to upgrade it and how to make use of container virtualization there. I~focused on possible problems such as creating the Docker registry, how to deal with the security and contents of images, how to log and where to store logs for further analysis and I~also had to have in mind that containers are able to move between machines but logs have to be processed correctly. I~had to propose how to monitor the applications in the containers and also the containers themselves. And I~had to find out how to deploy the static content of the webpages using load balancers.

I~successfully dealt with all of those issues and after consulting with our security administrators I~suggested how to secure and monitor what is in the images that we are running. I~suggested the architecture of the Docker registry we have to build. This architecture achieved high availability thanks to the Swift storage on which the Docker registry will be running in each data centre. I~explored different network management solutions that are possible to use in Kubernetes and chose to start the cluster with the flannel. The flannel has a significant overhead compared to other possibilities but it is sufficient for the purposes of my thesis.

It is important to know how the new Kubernetes cluster is behaving under heavy load and in other specific situations such as deadlocked applications, too many open files or network connections, so I~developed an application for testing such cluster. This application uses secrets with SSL certificates, secrets with configuration and produces logs. It also uses a persistent storage, for which I~utilized GlusterFS servers. The application provides Prometheus metrics on its service interface. Moreover I~created another application that watches a specified directory and sends logs to kafka for further processing. I~had to resolve many problems with reliability and performance and I~handled them successfully and implemented the Kafkafeeder application which will be present in each pod for uploading the logs.

In the end I~focused on better repeatability of the cluster creation procedure and I~wrote scripts that simplify installing it again later or adding new nodes to the existing cluster. I~wrote the Kubernetes configurations for my applications, creating services, secrets and other prescriptions used in Kubernetes.

During my work on this thesis I~found answers for many generic issues associated with starting the Kubernetes cluster while only a few of them were specific for Seznam.cz, so I~decided to open source all my code and images I~created. Those applications, images and the Kubernetes configurations can be used as examples for further development and can help others as well.
