\chapter{Introduction}

Given the performance of today’s servers, it is almost impossible to use their full potential with a single application. That’s where the possibility to install more than one application on each server comes to mind. However, this technique poses certain risks. What about when one application needs specific libraries or a whole environment in a certain version and another one needs something different? That’s exactly when virtualization fits. Using virtualization we can run many applications on a single machine without them knowing about each other.

Virtualization does not solve all problems and issues. One of the main problems with virtualization servers is a very difficult scaling as there is no easy way to automatically react on application’s needs. Simple example: Let’s have a website and create a virtual environment for it which we deploy on 2 servers (because of backup in case of one machine’s failure). Together with this application there can be many others running in their virtual spaces. When traffic rises to unexpected levels,  the application can quickly demand more resources than the virtual machine can provide and as there are more virtuals on the master, resource allocation cannot be increased. So the whole virtual have to be moved to different machine which has more resources available. And here comes the looking for it. Looking for machine with more resources can be really hard task, because there is also possibility that no other machine have enough resources for this kind of application. So someone have to decided which application can be moved on which machines to make a space for this, actually greedy one. But this situation may occur again few hours later, when this application will not need as much and other one will be busy.

Problems like this -- and many others -- can be solved using Kubernetes \cite{kubernetesio}. Kubernetes is an open-source system for automating deployment, operation, and scaling of containerized applications. Application and its virtual environment has to be packed in a container. In case of Kubernetes, the recommended container technologiy is Docker. Kubernetes automatically starts containers on physical machines in as many instances as the maintainer defines. It is very simple to automatically react on each application’s needs, give them more resources, move them among machines, scale them up and down and run auxiliary jobs at night when the flow of many applications is smaller to use the server house more efficiently.

Kubernetes offers all those possibilities and that’s why I chose to examine it more closely and start a testing instance of a private cloud based on Kubernetes at Seznam.cz company.