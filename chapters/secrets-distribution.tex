\chapter{Secrets distribution}

Kubernetes has a mechanism for storing secrets –- passwords, keys, certificates, etc. Unsurprisingly, they call it the secret.

Objects of the type secret are intended to hold sensitive information, such as passwords, OAuth tokens, and ssh keys. Putting this information in a secret is safer and more flexible than putting it verbatim in a pod definition or in a Docker image. \cite{secrets} 

Using secrets is safer than putting those sensitive information somewhere else, like directly to Docker images, to the file system mounted as a volume, to environment variables, etc. The main idea of the secret is to keep them centralized somewhere safe and distribute them only to those master nodes which need them. And also of course to have them on the master node only as long as necessary. For those purposes Kubernetes are using etcd as a persistent centralized storage that ensures high availability through peer-to-peer synchronization between machines. Each master node has a Kubelet daemon which can ask secrets API server for a secret. Communication from the master node to the etcd is encrypted. The Etcd sends the secret object to the master as a base64 encoded string. The size of the secret is limited to 1 MB and the master node will save the secret value to its memory, Not to the file system, which means that when the master node crashes and its memory is deleted, no secret can be compromised. The Kubelet then mounts the secret value to containers in the pod which requested it as a tmpfs filesystem. Secrets in the etcd are divided into namespaces and only pods from the same namespace can ask for them. It is the main responsibility of administrators to check the pod’s namespace when they accept it and deploy it to a running cluster.

After consulting the Kubernetes secret with our administrators we agreed that it would be nice to have the same feature for the current virtualization technologies (LXC and OpenVZ containers). The logic can be almost the same as Kubernetes have. And because the Kubelet is a standalone binary which can be called even without Kubernetes and the etcd is also an independent component, we decided to build a secrets distribution system for the current solutions at Seznam.cz. Each LXC container or OpenVZ virtual machine will have the Kubelet binary in its image. After installing a new virtual machine, the administrator will add a certificate signed by his team’s certification authority, which will be configured at etcd so it can request only secrets belonging to his team. There can be special one way certificates for highly sensitive data as well. After the virtual machine starts, it will start monitoring a known file, such as \lstinline{/www/secrets/request.json}, where applications or even the administrator of the virtual machine have to specify which secrets are requested. The Kubelet then asks the etcd for them and saves them via tmpfs to a known place such as \lstinline{/www/secrets/data/}. New applications have to be edited for those new features.

The main advantages of this approach is that all secrets are saved at one safe plac with a high availability for the whole company infrastructure and both master and virtual machines do not keep secrets for ever but only on a temporary file system in their memory.
