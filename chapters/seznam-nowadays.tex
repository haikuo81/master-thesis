\chapter{Seznam.cz nowadays}

Seznam.cz is a big company with more than a thousand employees. More than one quarter of them are developers. Seznam.cz is divided into compartments and developers in each compartment are grouped to teams led by team leaders. Each team leader is responsible for applications allocated to his team. There are a couple of recommendations on workflow (e.g. unified coding style),  but as different teams have different demands and use cases for their applications, they also have a slightly different requirements on coding, building, testing, deployment etc.

All over the company we are using Git as a version control system. Git is installed centrally using GitLab as a management system. GitLab provides CI – Continuous integration \cite{gitlabci}.
So each team has a possibility to easily run automatic builds or push to a repository after a merge request.

Testing and developing application is each team’s responsibility. But when someone wants to deploy an app to a production environment, there is a strict process. Each developer has access rights to the core machine with all the Debian \cite{debian} repositories. Developers have to upload Debian packages to the right branch (there are development, unstable, testing and read-only stable branches for each Debian distribution, divided per compartments), and run an incremental build of the branch.

We are using a request tracker system. Developers have to write a request consisting of packages which have to be moved to the stable branch, steps to install and run them on each server and list of servers affected by this change. A~process to rollback has to be described too.

An administrator from the team which manages the affected servers then takes the ticket and starts deploying. He can consult the process with the developer, who then tests the application to check whether the deployment was successful and then traffic is renewed (if it was stopped before) and the installation continues on the next server in the list.

This means developers have to turn anything they want to deploy to Debian packages, which depend on other packages. It is quite a simple but an efficient way to manage dependencies, to list software installed on each server or to see who is responsible for changes and maintainance of the package.

The problem with this process arises when you want to use anything that does not have a Debian package. Developers than have to create one from an upstream source code (that is not a problem) and maintain it so that it doesn’t become obsolete after a while – and that is problem.

The same situation can happen when developer wants to use a package from pip or an updated version of a package, that wasn’t updated in the official Debian repository. Adding such packages to the tree leads to a massive growth of the repository and a number of patched versions of software.
